% BASED ON SIGPROC-SP.TEX - V3.1SP - APRIL 2009
% WORKS WITH V3.2SP OF ACM_PROC_ARTICLE-SP.CLS

\documentclass{acm_proc_article-sp}

\usepackage{hyperref}
\hypersetup{pdfborder=0 0 0}

\begin{document}

\title{Browsix: COMPSCI 630 Project 1}
\subtitle{A UNIX-like process-model and kernel for the browser}

\numberofauthors{2}
\author{
\alignauthor Bobby Powers\\
       \email{bobbypowers@gmail.com}
\alignauthor Craig Greenberg\\
       \email{q7h0u6h7@gmail.com}
}
\date{23 Oct 2015}

\maketitle
\begin{abstract}
  This paper introduces Browsix, a UNIX-like processing model and
  kernel designed to run in modern web browsers.  The approach, system
  design, and preliminary results are described herein.
\end{abstract}

\section{Introduction}

As noted by Vilk et. al, web browsers have become a \emph{de facto}
universal operating system and attractive platform for application
developers\cite{vilk:2014doppio}.  Web browsers continue to expand the
range of functionality they provide to web applications through
Javascript APIs, including primitive cooperative
multitasking\cite{mcilroy:2015chrome47} and low-level hardware
interaction like Bluetooth\cite{yasskin:2015webbluetooth}.  Javascript
and its event-based programming model have become popular outside of
web browsers - node.js being the canonical example.  Empirically, node.js
enables the creation of high performance web servers by pairing the V8
Javascript engine with asynchronous APIs.

This project has several contributions:

\begin{itemize}
  \item An implementation of a traditional UNIX-like kernel,
    userspace, and syscall abstractions in the browser, utilizing the
    Web Workers API.  We refer to this as the process model.
  \item A port of the node.js programming environment to the browser
    on top of our process model.
  \item A collection of traditional UNIX utilities, implemented as
    node.js applications in TypeScript.  These utilities run
    unmodified in both our browser environment and under node.js on
    Mac OS X and Linux.
  \item A bash-like shell that enables the composition of utilities
    into pipelines (sometimes called filters).
  \item A web-based UI for interacting with this shell that works in
    all modern browsers.
\end{itemize}


Souci and Lemaire\cite{souci:2014} describe node's architecture at a
high level.

\section{Approach}

\section{System Design}

\section{Results}

\section{Discussion}

\section{Conclusion}

\bibliographystyle{abbrv}
\bibliography{report}

\balancecolumns
\end{document}
